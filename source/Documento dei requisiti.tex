\documentclass[a4paper,12pt]{article}

%%%%%%%%%%%%%%%%%%%%
%%%%  PREAMBLE  %%%%
%%%%%%%%%%%%%%%%%%%%

\usepackage[T1]{fontenc}
\usepackage[utf8]{inputenc}

\usepackage[english,italian]{babel}

\usepackage{hyperref}
\hypersetup{hidelinks}

\usepackage[margin=2.5cm]{geometry}
\usepackage{minipage-marginpar}
\usepackage{fancyhdr}
\usepackage[bottom]{footmisc}
\usepackage{lastpage}

\usepackage{enumitem}

\usepackage{graphicx}

\setlength{\parindent}{0em}
\setlength{\parskip}{1em}

\fancyhead[L]{\leftmark}
\fancyhead[R]{\shortstack[r]{Versione documento: 0.01 \\ Gruppo: T27}}

\fancyfoot[C]{}
\fancyfoot[R]{\thepage/\pageref{LastPage}}

\renewcommand{\headrulewidth}{2pt}
\renewcommand{\headruleskip}{3pt}
\setlength{\headheight}{30pt}

\renewcommand{\footrulewidth}{2pt}

\setlist[itemize]{itemsep=0.25em,topsep=0pt}
\setlist[enumerate]{itemsep=0.75em,topsep=0pt,align=left}

%%%%%%%%%%%%%%%%%%%%
%%%%  DOCUMENT  %%%%
%%%%%%%%%%%%%%%%%%%%

\title{Web Music Player}
\author{Gruppo T27}

\begin{document}

\pagestyle{empty}

\begin{center}

    \vspace{2 cm}

    \begin{tabular*}{\textwidth}{ c @{\extracolsep{\fill}} c }
        \includegraphics[width=0.3\textwidth]{marchio_unitrento.pdf} & \shortstack{\Large{Dipartimento di Ingegneria} \\ \Large{e Scienza dell'Informazione}}
    \end{tabular*}

    \vspace{2 cm} 
  
    \LARGE{Ingegneria del software\\}
  
    \vspace{1.5 cm} 
    \Large\textsc{Documento dei requisiti\\} 
    \Large\textsc{Versione: 0.01\\} 
    \vspace{2 cm} 
    \Huge\textsc{Web Music Player\\}
    \Large{\it{Gruppo T27}}
  
    \vspace{2 cm} 
  
    \Large{Anno accademico 2022/2023}
\end{center}

\newpage
\tableofcontents

\pagestyle{fancy}

\newpage
\section{Scopo del documento}

Il presente documento riporta l’analisi dei requisiti di sistema del progetto Web Music Player. Lo scopo di questo di questo documento è quello di:
\begin{itemize}
    \item descrivere gli obiettivi funzionali;
    \item elencare i requisiti non funzionali;
    \item mostrare le interazioni del progetti con altri sistemi;
    \item mostrare le componenti interne del sistema.
\end{itemize}

\newpage
\section{Requisiti funzionali}

\newpage
\section{Requisiti non funzionali}

% TODO: INTRODUZIONE

\newpage
\section{Diagramma di contesto}

\newpage
\section{Diagramma delle componenti}

\end{document}