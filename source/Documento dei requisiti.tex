\documentclass[a4paper,12pt]{article}

%%%%%%%%%%%%%%%%%%%%
%%%%  PREAMBLE  %%%%
%%%%%%%%%%%%%%%%%%%%

\usepackage[T1]{fontenc}
\usepackage[utf8]{inputenc}

\usepackage[english,italian]{babel}

\usepackage{hyperref}
\hypersetup{hidelinks}

\usepackage[margin=2.5cm]{geometry}
\usepackage{minipage-marginpar}
\usepackage{fancyhdr}
\usepackage[bottom]{footmisc}
\usepackage{lastpage}

\usepackage{enumitem}

\usepackage{graphicx}

\setlength{\parindent}{0em}
\setlength{\parskip}{1em}

\fancyhead[L]{\leftmark}
\fancyhead[R]{\shortstack[r]{Versione documento: 0.01 \\ Gruppo: T27}}

\fancyfoot[C]{}
\fancyfoot[R]{\thepage/\pageref{LastPage}}

\renewcommand{\headrulewidth}{2pt}
\renewcommand{\headruleskip}{3pt}
\setlength{\headheight}{30pt}

\renewcommand{\footrulewidth}{2pt}

\setlist[itemize]{itemsep=0.25em,topsep=0pt}
\setlist[enumerate]{itemsep=0.25em,topsep=0pt,align=left}

%%%%%%%%%%%%%%%%%%%%
%%%%  DOCUMENT  %%%%
%%%%%%%%%%%%%%%%%%%%

\title{Web Music Player}
\author{Gruppo T27}

\begin{document}

\pagestyle{empty}

\begin{center}

    \vspace{2 cm}

    \begin{tabular*}{\textwidth}{ c @{\extracolsep{\fill}} c }
        \includegraphics[width=0.3\textwidth]{marchio_unitrento.pdf} & \shortstack{\Large{Dipartimento di Ingegneria} \\ \Large{e Scienza dell'Informazione}}
    \end{tabular*}

    \vspace{2 cm} 
  
    \LARGE{Ingegneria del software\\}
  
    \vspace{1.5 cm} 
    \Large\textsc{Documento dei requisiti\\} 
    \Large\textsc{Versione: 0.01\\} 
    \vspace{2 cm} 
    \Huge\textsc{Web Music Player\\}
    \Large{\it{Gruppo T27}}
  
    \vspace{2 cm} 
  
    \Large{Anno accademico 2022/2023}
\end{center}

\newpage
\tableofcontents

\pagestyle{fancy}

\newpage
\section{Scopo del documento}

Il presente documento riporta l’analisi dei requisiti di sistema del progetto Web Music Player. Lo scopo di questo di questo documento è quello di:
\begin{itemize}
    \item descrivere gli obiettivi funzionali;
    \item elencare i requisiti non funzionali;
    \item mostrare le interazioni del progetti con altri sistemi;
    \item mostrare le componenti interne del sistema.
\end{itemize}

\newpage
\section{Requisiti funzionali}

Andiamo a descrivere nel dettaglio i requisiti funzionali del sistema. Per fare ciò, sfruttiamo gli use-case diagram, un utile strumento di visualizzazione. Ciascuno use-case diagram sarà accompagnato da una descrizione; questa può essere testuale o tramite un ulteriore diagramma.

\subsection*{RF1 Registrazione}

\begin{figure}[htp]
    \centering
    \includegraphics[width=0.75\textwidth]{diagrams/use-case-1.png}
\end{figure}

Per descrivere questo use-case, facciamo uso di un diagramma delle attività:

\begin{figure}[htp]
    \centering
    \includegraphics[width=0.75\textwidth]{diagrams/activity-1.png}
\end{figure}

\newpage

\subsection*{RF2 Pagamento}

\begin{figure}[htp]
    \centering
    \includegraphics[width=0.75\textwidth]{diagrams/use-case-2.png}
\end{figure}

\textbf{Descrizione:} l'utente paga il servizio tramite un sistema esterno. 

\vspace{1em}
\subsection*{RF3 Login}

\begin{figure}[htp]
    \centering
    \includegraphics[width=0.75\textwidth]{diagrams/use-case-3.png}
\end{figure}

\textbf{Descrizione:} questo use-case descrivere l’accesso al sito per un utente già registrato.

\textbf{Passi:}
\begin{enumerate}
    \item L’utente inserisce la sua mail 
    \item L’utente inserisce la password associata al suo account 
    \item L’utente conferma la mail e la password inserite
    \item Il sistema verifica la correttezza di mail e password \textbf{[exception 1]}
    \item L’utente entra nel sito
\end{enumerate}

\textbf{[exception 1]} Nel caso in cui l’email e/o la password siano errati viene esposto un messaggio di errore: “email e/o password errati”.

\newpage

\subsection*{RF4 Recupero password}

\begin{figure}[htp]
    \centering
    \includegraphics[width=0.75\textwidth]{diagrams/use-case-4.png}
\end{figure}

Per descrivere questo use-case, facciamo uso di un diagramma delle attività:

\begin{figure}[htp]
    \centering
    \includegraphics[width=0.75\textwidth]{diagrams/activity-4.png}
\end{figure}

\subsection*{RF 5-13}

\begin{figure}[htp]
    \centering
    \includegraphics[width=0.75\textwidth]{diagrams/use-case-5-6-7-8-9-10-11-12-13.png}
\end{figure}

\subsubsection*{RF5 Ricerca}
\textbf{Descrizione:} l’utente deve poter ricercare tra releases e le proprie playlists. \newline
\textbf{Passi:}
\begin{enumerate}
    \item L’utente inserisce nella barra di ricerca del testo
    \item Il sistema ricerca in modo approssimato tra le releases caricate e le playlist e restituisce una lista di risultati, ordinati in base a quello più vicino alla query
\end{enumerate}

\subsubsection*{RF 6 Riproduzione}

\textbf{Descrizione:} l’utente deve poter riprodurre una release o una playlist a partire da un brano e avere quelli successivi automaticamente messi nella coda di riproduzione nel corretto ordine. \newline
\textbf{Passi:}
\begin{enumerate}
    \item L’utente preme sull’apposito pulsante per iniziare la riproduzione del brano \textbf{[extension 1]}
    \item Il Client dialoga col sistema per iniziare lo stream
    \item Il sistema inoltre aggiunge la playlist o release alla cronologia
\end{enumerate}
\textbf{[extension 1]} Se il brano si trova in una playlist e non è l’ultimo, il client si dovrà premurare di aggiungere anche i brani che vengono dopo alla coda nello stesso ordine in cui appaiono nella playlist. Altrimenti, se il brano viene riprodotto come parte di una release e non è l’ultimo, il client dovrà aggiungere alla coda i brani che succedono quello riprodotto nell’ordine corretto.

\subsubsection*{RF 7 Coda}

\textbf{Descrizione:} l’utente deve poter gestire la propria coda di riproduzione (riordinarla, rimuovervi brani e aggiungerne) e il client deve seguire la coda per sapere l’ordine in cui vanno riprodotte le canzoni. \newline
\textbf{Passi:}
\begin{enumerate}
    \item L’utente, attraverso appositi pulsanti, è in grado di riordinare, rimuovere e aggiungere brani alla coda
    \item Il Client, unico posto dove la coda è mantenuta, deve salvare ogni modifica
    \item Una volta che la riproduzione del brano corrente è terminata, la coda va utilizzata per trovare il prossimo brano da riprodurre
\end{enumerate}

\subsubsection*{RF 8 Libreria}

\textbf{Descrizione:} l’utente deve poter vedere le proprie playlists e la musica che ha salvat.

\subsubsection*{RF 9 Musica che ti piace}

\textbf{Descrizione:} l’utente deve poter salvare le releases che ascolta in una raccolta speciale chiamata “Musica che ti Piace” e successivamente deve essere in grado di gestire (rimuovere, riordinare o aggiungere releases) suddetta raccolta. \newline
\textbf{Passi:}
\begin{enumerate}
    \item Durante la riproduzione di un brano o altri momenti, l’utente potrà, attraverso un apposito pulsante, chiedere al sistema di aggiungere una release alla raccolta “Musica che ti piace” dell’utente
    \item Successivamente, da apposite tab, è possibile, utilizzando appositi pulsanti, dialogare col sistema per riordinare o rimuovere releases dalla raccolta
\end{enumerate}

\subsubsection*{RF 10 Playlist}

\textbf{Descrizione:} l’utente deve poter salvare i brani che ascolta in raccolte chiamate playlists e successivamente deve essere in grado di gestirle (rimuovere, riordinare o aggiungere releases e rimuovere playlists). \newline
\textbf{Passi:}
\begin{enumerate}
   \item Durante la riproduzione di un brano o altri momenti, l’utente potrà, attraverso un apposito pulsante, chiedere al sistema di aggiungere un brano a una delle playlists che ha già creato o in una nuova, a cui deve dare un nome
    \item Successivamente, da apposite tab, è possibile, utilizzando appositi pulsanti, dialogare col sistema per riordinare o rimuovere le releases di una playlist o per rimuovere interamente una playlist
\end{enumerate}

\subsubsection*{RF 11 Consigliati}

\textbf{Descrizione:} Il sistema suggerirà all’utente releases dal proprio database, con l'obiettivo di consigliargli brani che possano piacergli.

\subsubsection*{RF 12 Feedback}

\textbf{Descrizione:} l’utente deve essere in grado di poter restituire al sistema feedback riguardo i consigli, per aiutarlo nel perfezionamento di futuri consigli. \newline
\textbf{Passi:}
\begin{enumerate}
   \item L’utente si vedrà arrivare dei consigli dal sistema, grazie agli appositi pulsanti sarà in grado di riprodurre la release consigliata e segnalare al sistema se questa era di suo gradimento o meno
\end{enumerate}

\subsubsection*{RF 13 Cronologia}

\textbf{Descrizione:} l’utente deve poter recuperare la lista delle ultime 30 riproduzioni (siano esse releases o playlist). \newline
\textbf{Passi:}
\begin{enumerate}
    \item L’utente si reca sul menù dedicato alla cronologia
    \item Il sistema recupera la lista delle ultime 30 riproduzioni e le mostra all’utente
\end{enumerate}

\subsection*{RF 14-17}

\begin{figure}[htp]
    \centering
    \includegraphics[width=0.75\textwidth]{diagrams/use-case-14-15-16-17.png}
\end{figure}

\subsubsection*{RF 14 Impostazioni}

\textbf{Descrizione:} l’utente può accedere alla pagina delle impostazioni, dalla quale può compiere una serie di azioni relative al proprio account e alla piattaforma.

\subsubsection*{RF 15 Logout}

\textbf{Descrizione:} l’utente può disconnettersi dalla piattaforma. \newline
\textbf{Passi:}
\begin{enumerate}
    \item L’utente si reca alla pagina delle impostazioni
    \item L’utente preme un pulsante per effettuare la disconnessione del proprio account
    \item L’utente viene riportato alla pagina di login
\end{enumerate}

\subsubsection*{RF 16 Disdire l'abbonamento}

\textbf{Descrizione:} l’utente può disconnettersi dalla piattaforma. \newline
\textbf{Passi:}
\begin{enumerate}
    \item l’utente si reca alla pagina delle impostazioni
    \item l’utente preme un pulsante per terminare l’abbonamento alla piattaforma
    \item l’utente potrà usufruire del servizio fino al termine del mese per il quale ha pagato
    \item una volta terminato quel periodo, dopo l’accesso alla piattaforma l’utente sarà portato alla pagina del pagamento
\end{enumerate}

\subsubsection*{RF 17 Cambio lingua}

\textbf{Descrizione:}  l’utente può cambiare la lingua della piattaforma. \newline
\textbf{Passi:}
\begin{enumerate}
    \item l’utente si reca alla pagina delle impostazioni
    \item l’utente tramite un menù a tendina può scegliere la lingua di visualizzazione della piattaforma
    \item il servizio si adatta, modificando i vari campi testuali per riflettere la lingua selezionata dall’utente
\end{enumerate}


\subsection*{RF 18-19}

\begin{figure}[htp]
    \centering
    \includegraphics[width=0.75\textwidth]{diagrams/use-case-18-19.png}
\end{figure}

\subsubsection*{RF 18 Account creator}

\textbf{Descrizione:}  l’account creator consiste in un'estensione dell'account standard.

\subsubsection*{RF 19 Visualizzare account creator}

\textbf{Descrizione:} gli utenti creator e gli utenti standard possono visualizzare le pagine degli account creator. Qui sono presenti due elenchi: i singoli caricati e gli album caricati da quel creator.

\subsection*{RF20 Caricare releases}

\begin{figure}[htp]
    \centering
    \includegraphics[width=0.75\textwidth]{diagrams/use-case-20.png}
\end{figure}

\textbf{Descrizione:} questo use case descrive il caricamento delle releases sulla piattaforma da parte dei creator.

\textbf{Passi:}
\begin{enumerate}
    \item l’utente creator sceglie di caricare una release dalla sua pagina
    \item l’utente creator seleziona la release da caricare tramite il browser
    \item l’utente creator sceglie il titolo da assegnare alla release \textbf{[extension 1]}
    \item l’utente creator conferma la scelta
    \item la release viene salvata sul database della piattaforma
\end{enumerate}
\textbf{[extension 1]} l’utente creator può aggiungere dei tag alla release caricata.

\subsection*{RF21 Modificare releases}

\begin{figure}[htp]
    \centering
    \includegraphics[width=0.75\textwidth]{diagrams/use-case-21.png}
\end{figure}

\textbf{Descrizione:} gli utenti creator possono modificare il titolo e i tag delle releases che hanno caricato sulla piattaforma. Questo dalla schermata del loro account.

\textbf{Passi:}
\begin{enumerate}
    \item l’utente creator seleziona una release dalla propria pagina
    \item l'utente sceglie l’opzione di modifica
    \item l'utente può da qui modificare il titolo, i tag, o entrambi
    \item le modifiche vengono salvate nel database
    
\end{enumerate}

\subsection*{RF22 Eliminare releases}

\begin{figure}[htp]
    \centering
    \includegraphics[width=0.75\textwidth]{diagrams/use-case-22.png}
\end{figure}

\textbf{Descrizione:} gli utenti creator possono cancellare dalla piattaforma releases da loro caricate. Questo dalla schermata del loro account.

\textbf{Passi:}
\begin{enumerate}
    \item l’utente creator seleziona una release dalla propria pagina
    \item l'utente sceglie l’opzione di eliminazione
    \item l'utente conferma l’eliminazione della release dalla piattaforma
    \item la release viene rimossa dal database    
\end{enumerate}

\newpage
\section{Requisiti non funzionali}

% TODO: INTRODUZIONE

\newpage
\section{Diagramma di contesto}

\newpage
\section{Diagramma delle componenti}

\end{document}